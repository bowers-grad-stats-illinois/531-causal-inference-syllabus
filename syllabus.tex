\documentclass[11pt,letterpaper]{article}

% Modern packages for styling
\usepackage[utf8]{inputenc}
\usepackage[T1]{fontenc}
\usepackage{helvet} % Helvetica font (sans serif)
\renewcommand{\familydefault}{\sfdefault} % Set sans serif as default
\usepackage[margin=1in]{geometry}
\usepackage{xcolor}
\usepackage{fontawesome5}
\usepackage{hyperref}
\usepackage{enumitem}
\usepackage{titlesec}
\usepackage{fancyhdr}
\usepackage{datetime}
\usepackage{advdate}

% Bibliography setup
\usepackage[style=authoryear, backend=biber, maxbibnames=10, maxcitenames=2]{biblatex}
\addbibresource{syllabus.bib}

% Define color scheme
\definecolor{primarycolor}{RGB}{19, 41, 75} % Illinois Navy Blue
\definecolor{accentcolor}{RGB}{232, 74, 39} % Illinois Orange
\definecolor{lightgray}{RGB}{100, 100, 100}

% Hyperlink setup
\hypersetup{
    colorlinks=true,
    linkcolor=primarycolor,
    urlcolor=accentcolor,
    citecolor=primarycolor
}

% Section title formatting
\titleformat{\section}
{\color{primarycolor}\Large\bfseries}
{\thesection}{1em}{}[\titlerule]

\titleformat{\subsection}
{\color{accentcolor}\large\bfseries}
{\thesubsection}{1em}{}

% Header and footer
\pagestyle{fancy}
\fancyhf{}
\fancyhead[L]{\color{lightgray}\small POLS 531: Causal Inference}
\fancyhead[R]{\color{lightgray}\small Spring 2026}
\fancyfoot[C]{\thepage}
\fancyfoot[R]{\footnotesize{Version~of~\today~\currenttime}}
\renewcommand{\headrulewidth}{0.4pt}
\renewcommand{\footrulewidth}{0pt}

% Date calculation setup
% CUSTOMIZE: Change this date to adjust the first day of class
\newcommand{\firstclassdate}{20/01/2026}

\newdateformat{classformat}{\THEDAY\ \monthname[\THEMONTH] \THEYEAR}

% Counter for automatic week numbering
\newcounter{weeknum}
\setcounter{weeknum}{0}

% \classweek{offset} - Prints the date that is 'offset' weeks from firstclassdate
% Example: \classweek{0} = first class, \classweek{1} = second week, \classweek{9} = week after spring break
% This allows you to:
%   - Add/remove weeks by changing just one number
%   - Insert spring break by skipping a week number (e.g., go from \classweek{7} to \classweek{9})
%   - Reorder weeks without affecting other dates
\newcommand{\classweek}[1]{%
  \stepcounter{weeknum}%
  \SetDate[\firstclassdate]%
  \AdvanceDate[\the\numexpr#1*7\relax]%
  \classformat\today%
}

% Legacy command for backwards compatibility (auto-incrementing)
% \weekdate - Automatically advances by 7 days each call
\newcounter{autoweek}
\setcounter{autoweek}{-1}
\newcommand{\weekdate}{%
  \stepcounter{autoweek}%
  \classweek{\theautoweek}%
}

\usepackage{parskip}

\begin{document}

% Custom title
\begin{center}
  {\Large\color{accentcolor}\bfseries Causal Inference for Political Science}\\
  {\Large\color{primarycolor}\bfseries POLS 531}\\
  {\large Spring 2026}\\
  {\normalsize University of Illinois at Urbana-Champaign}
  \rule{\textwidth}{1pt}
\end{center}

\section*{\faInfoCircle\ Course Information}

\noindent
\begin{tabular}{@{}ll}
\textbf{\faCalendar\ Meeting Time:} & Tuesday, 9:00 AM -- 11:50 AM \\
\textbf{\faMapMarker\ Location:} & David Kinley Hall 314 \\
\textbf{\faUser\ Instructor:} & Jake Bowers \\
\textbf{\faEnvelope\ Instructor Email:} & \href{mailto:jwbowers@illinois.edu}{jwbowers@illinois.edu} \\
\textbf{\faClock\ Office Hours:} & by appointment at \href{https://calendly.com/jakebowers}{Calendly}\\
\textbf{\faBuilding\ Office:} & 432 David Kinley Hall \\
    \textbf{\faUser\ Methods Preceptor:} & Liliana Brock \\
    \textbf{\faEnvelope\ Methods Preceptor Email:} & \href{lcbrock2@illinois.edu}{lcbrock2@illinois.edu}
\end{tabular}

\section*{\faBook\ Overview}

We infer what we cannot observe. Last term you engaged with the problems of
choosing better or worse descriptions of what we actually observe in data, and
also with the problems of statistical inference --- where we discussed better
and worse ways to learn about descriptions of phenomenon that we cannot
directly observe using tools like estimators and tests. Whether an estimator or
a test was a good one or bad one depended on assumptions that we had to make
about the process of observation. For example, we learned that a randomized
experiment and/or a random sample allows us to trust that our estimators are
unbiased and tests have controlled false positive error rates.

This term we dive deeper in the different approaches that statisticians and
methodologists have developed to make causal inferences.\footnote{I encourage
you to check out classes on sampling in order to go deeper into population
inference and classes on measurement and psychometrics in order to go deeper
into measurement inference.} When we say "causal inference" in this course we
are really talking about making statistical inferences that we feel comfortable
interpreting as reflecting counterfactual causal effects.\footnote{We recommend
you checkout \textcite{waldner2025book} to learn more about causal inference that
does not involve statistical inference directly.}

What we will see is that ``statistical inference for counterfactual causal
quantities" aka ``causal inference" requires more assumptions than statistical
inference alone. The fact that we have to make assumptions is not in and of
itself a problem. However, more assumptions require more scrutiny. By the end
of the course, I hope that you feel confident choosing, using and scrutinizing
assumptions to enable you to make causal inferences from data and designs that
help you advance understanding of politics and society in your substantive
work.

\section*{\faBullseye\ Learning Objectives}

By the end of this course, students will be able to:

\begin{itemize}[leftmargin=*]

  \item Articulate the fundamental problem of causal inference and distinguish
    causal from descriptive research questions

  \item Design and analyze basic randomized experiments, including power
    analysis.

  \item Understand, implement, interpret, and evaluate causal inferences that
    depend on adjusting for confounders (including matching and parametric
    adjustment)
  
  \item Understand, implement, interpret, and evaluate causal inferences that
    depend on instrumental variables and other discontinuities (RDD)

  \item Understand, implement, interpret, and evaluate causal inferences that
    depend on assumptions about how outcomes change over time in panel data.

\end{itemize}

    To these ends I have designed a series of activities that should (1)
    give you opportunities to practice working with data and reasoning
    about statistics and (2) raise questions for discussion each week.

    \textbf{I do not lecture.} Rather, each week we will meet to engage with the
    questions that you have.

    \subsection{Explorations}

    Every week or so, I will ask you to complete a short assignment that
    encourages you to engage creatively with the topics of that section of the
    course. I anticipate that you will work on most of these assignments in
    groups and a few alone  and that each of you will come to class prepared to
    discuss them.  I don't think that the groups should have more than 3 people
    in them. However, I'm willing to have larger groups if you talk with me
    about it. The point of the explorations is for you to (1) practice learning
    on your own (making mistakes, confronting confusing error messages, finding
    help online and elsewhere) and in a group (this is how you will learn about
    statistics for the rest of your career, so these explorations are supposed
    to help you to practice it now), (2) engage with the topic of the week so
    that you are prepared to come to class with questions and ideas, (3)
    practice coding and confronting coding errors.

    \subsection{Final projects}

    Each of you will produce a short replication paper by the end of the
    term.\footnote{The idea comes from
        \href{https://gking.harvard.edu/papers}{Gary King's assignment to his
        first year PhD students}. I encourage you read that website and
        associated paper in which King explains his ideas. Our version will differ
    a little from his.} The idea is to practice using your new skills and concepts
    as applied to a topic of interest to you --- but in a very targeted way. The
    idea is to understand what someone else did in the past, and perhaps to improve
    upon it now.

    You should find a paper that uses statistical methods that you are willing
    to work to understand this term and where the data are available (ideally
    the data are easy to download, you can also contact the author of the paper
    after discussion with me).

    You are allowed to do this work with a co-author or alone as you see fit.

    I will be providing more detailed guidance about this assignment as the
    term goes on.

    \subsection{My Expectations}

    \begin{enumerate}
        \item I assume you are eager to learn. Eagerness,
            curiosity and excitement will impel your energetic engagement with the
            class throughout the term. If you are bored, not curious, or unhappy
            about the class you should come and talk with me
            immediately. Energetic engagement manifests itself in meeting with
            your classmates outside of the class, in asking questions during the
            class, and in taking the assignments seriously.

        \item I assume you are ready to work. Learning requires work. As much as
            possible I will encourage you to link practice directly to application
            rather than merely as a opportunity for me to rank you among your peers.
            Making work about learning rather than ranking, however, will make our work
            that much more difficult and time consuming. You will make errors. These
            errors are opportunities for you to learn --- some of your learning will be
            about how to help yourself and some will be about statistics. If you have
            too much to do this term consider dropping the course. Graduate
            school is a place for you to develop and begin to pursue your own
            intellectual agendas: this course may be important for you this term, or it
            may not. That is up for you to decide.

        \item  I assume some previous engagement with high school mathematics.

        \item You should ask questions when you don't understand things;
            chances are you're not alone.  \textbf{This class is an opportunity
            to practice courage:} I expect you to make a guess when I ask a
            question (in writing or in person), I expect that you will ask a
            question when you have a problem.

        \item \textbf{Do the work.} This
            does not mean divide the work up among your classmates so that you only do
            part of the work. Each person should engage with all of the work even if the
            people who writes it up changes from week to week.

        \item 	All papers written in this class will assume familiarity with the
            principles of good writing in \textcite{beck:1986}.

        \item 	All final written work will be turned in as pdf files unless we have another
            specific arrangement.\footnote{For example, if you have some reason why pdf
                files make your life especially difficult, then of course I will work with
            you find another format.} I will not accept Microsoft, Apple, or any other
            proprietary format.
    \end{enumerate}

    \subsection{Late Work}
    I do not like evaluation for the sake of evaluation. Evaluation should
    provide opportunities for learning. So, if you'd prefer to spend
    more time using the paper assignment in this class to learn more, I am
    happy for you to take that time. I will not, however, entertain late
    submissions for any subsidiary paper assignments or other homeworks that are due
    throughout the term. If you think that you and/or the rest of the
    class have a compelling reason to change the due date on one of those
    assignments, let me know in advance and I will probably just change
    the due date for the whole class.

    \subsection{Incompletes} Incomplete grades at the end of the term are fine
    in theory but terrible in practice. I urge you to avoid an incomplete in
    this class. If you must take an incomplete, you must give me \emph{at
    least} 2 months from the time of turning in an incomplete before you can
    expect a grade from me and it may well take me much longer. This means that
    if your fellowship, immigration status, or job depends on erasing an
    incomplete in this class, you should not leave this incomplete until the
    last minute.

    \subsection{Grades are Feedback}

    Humans need feedback to close the gap between  intention and action. They
    also need feedback to feel good about their progress and to motivate them.
    In this class I will use grades as feedback. All grades except for the
    final grade will be satisfactory, unsatisfactory (with the possibility of
    "outstanding"), and fail. These map roughly onto  A+=outstanding,
    A=satisfactory, C=unsatisfactory, and F=fail (i.e. you didn't try).

    I'll calculate your grade for the course this way: 30\% daily R (you have 5
    days out of every 7 to turn it in, no late work accepted, satisfactory if
    you turned it in, fail if you didn't turn it in); 40\% explorations (when
    you turn it in as a group everyone in the group receives the same grade,
    satisfactory if you are creative and thoughtful and diligent,
    unsatisfactory if you are not or if you don't seem to be getting the
    concepts, no late work accepted); 20\% replication paper; 10\% attendance
    (satisfactory if you show up, fail if not).

    You can miss two classes without grade penalty.

    I will drop your lowest exploration grade as well.

    Because moments of evaluation are also moments of learning in this
    class, I do not curve. If you all perform at 100\%, then I will give
    you all As.

    You can redo any  evaluation or the final paper in order  to  increase
    your grade on that  assignment. If you want to resubmit something already
    graded, you need  to let me know  in advance so that  I  can make  time to
    grade it again. If you want to resubmit work after the end of the term,
    that is also ok, but I may take many months to grade that work.

    \subsection{Computing} We will be using R~in class so those of you with
    laptops available should bring them to class. Of course, I will not
    tolerate the use of computers for anything other than class related work
    during active class time. Please install R on your computers before the
    first class session.

    As you work on your papers, you will also learn to write about data
    analysis in a way that sounds and looks professional by using either
    R+markdown or Sweave (R+\LaTeX). No paper will be accepted without a code
    appendix or reproduction Github repository made available to me. No paper
    will be accepted unless it is in Portable Document Format
    (\href{http://en.wikipedia.org/wiki/Portable_Document_Format}{pdf}). No
    paper will be accepted with cut and pasted computer output in the place of
    well presented and replicable figures and tables. Although good empirical
    work requires that the analyst understand her tools, she must also think
    about how to communicate effectively: ability to reproduce past analyses
    and clean and clear presentations of data summaries are almost as important
    as clear writing in this regard.

    \section{Books}
    I'm am not requiring any particular books this term. The readings will be
    drawn from a variety of sources. I will try to make most of them
    available to you as we go if you can't find them easily online yourselves.

    \subsection{Recommended}
    No book is perfect for all students. I suggest you ask around, look at
    other syllabi online, and just browse the shelves at the library and
    used bookstores to find books that make things clear to you.  I will be
    adding some recommendations here. Let me know now if you have favorites.

    If you discover any books or websites that are particularly useful to you, please
    alert me and the rest of the class about them. Thanks!

    \subsection{Academic Integrity} According to the Student Code, `It is
    the responsibility of each student to refrain from infractions of
    academic integrity, from conduct that may lead to suspicion of such
    infractions, and from conduct that aids others in such infractions.’
    Please know that it is my responsibility as an instructor to uphold the
    academic integrity policy of the University, which can be found here:
    \url{http://studentcode.illinois.edu/article1_part4_1-401.html}.

\subsection*{Accessibility}
Students with disabilities who require accommodations should contact Disability Resources and Educational Services (DRES) and provide the instructor with accommodation letters as soon as possible.

\subsection*{Mental Health}
The University of Illinois offers a range of mental health services. If you are feeling overwhelmed, anxious, or depressed, please reach out to the Counseling Center (\href{https://www.counselingcenter.illinois.edu/}{counselingcenter.illinois.edu}) or call the Consultation and Crisis Line at 217-333-3704.

\subsection*{Diversity and Inclusion}
The Political Science Department is committed to building an inclusive environment where all students feel welcomed and supported. We value diverse perspectives and encourage respectful dialogue.

\section*{\faIcon{calendar-alt}\ Course Schedule}

    \textbf{Note: } This schedule is preliminary and subject to change. If
    you miss a class make sure you contact me or one of your colleagues to
    find out about changes.


    \textbf{Data: } I'll be bringing in data that I have on hand. This
    means our units of analysis will often be individual people or perhaps
    political or geographic units, mostly in the United States. I'd love
    to use other data, so feel free to suggest and provide it to me ---
    come to office hours and we can talk about how to use your favorite
    datasets in the class.

    % HOW TO USE: \classweek{N} prints the date N weeks after firstclassdate
% - Week numbers are offsets: 0 = first class, 1 = second week, etc.
% - To add spring break: skip a number (e.g., go from 7 to 9, leaving 8 as spring break)
% - To insert a new week: increment all subsequent week numbers by 1
% - To delete a week: decrement all subsequent week numbers by 1
% - To reorder: just swap the offset numbers

\subsection*{Week 1: \classweek{0} -- Introduction to the class, the Fundamental Problem of Causal Inference, Working with Data}
\begin{itemize}[leftmargin=*]
  \item The fundamental problem of causal inference and the potential outcomes framework
  \item Causal effects as functions of potential outcomes including individual causal effects and average treatment effects (ATE, ATT, ATC)
  \item Directed Acyclic Graphs (DAGs) to help us reason and communicate about causal relationships: the Backdoor criterion and colliders
  \item Workflow for credibility
\end{itemize}
\textbf{Required:} \textcite{holland1986}; \textcite{rosenbaum2017}, Ch. 1--2; \textcite{cunningham2021causal} Ch. 3--4; \textcite{bowers2016future}; \textcite{bowers2022latex} (no need to do the exercises)
\textbf{Recommended:} \textcite{leavittbowers2020}; \textcite{waldner2025book} Ch. 2 up through \S 2.2.2; TBA Glynn on front door adjustment.

\textbf{Due:}

\subsection*{Week 2: \classweek{1} -- Randomized Experiments I}
\begin{itemize}[leftmargin=*]
  \item Types of random assignment
  \item Fisher's exact test and randomization inference
\end{itemize}
\textbf{Required:} \textcite{rosenbaum2017}, Ch.\ 3; \textcite{fisher1935a}, \S 5--10, pp.\ 11--19.\\
\textbf{Recommended:} \textcite{rosenbaum2002a}, pp.\ 27--49; \textcite{rosenbaum2010}, Ch.\ 2.

\subsection*{Week 3: \classweek{2} -- Randomized Experiments II}
\begin{itemize}[leftmargin=*]
  \item Block randomization and stratification
  \item Cluster randomization
  \item Power analysis and sample size calculations
  \item Compliance and intention-to-treat analysis
\end{itemize}
\textbf{Required:} \textcite{gerbergreen2012}, Ch.\ 2--4; \textcite{lin2013}.\\
\textbf{Recommended:} \textcite{aronowmiddleton2013}; \textcite{middletonaronow2015}; \textcite{miratrixetal2013}.

\subsection*{Week 4: \classweek{3} -- Observational Studies and Selection Bias}
\begin{itemize}[leftmargin=*]
  \item Sources of bias in observational studies
  \item Selection on observables vs. unobservables
  \item Conditional independence assumption
  \item Introduction to confounding
\end{itemize}
\textbf{Required:} \textcite{bindrubin2019}; \textcite{gelmanhill2006}, \S 9.0--9.2; \textcite{berk2010}.\\
\textbf{Recommended:} \textcite{rosenbaum2017}, Ch.\ 5; \textcite{cochran1965}; \textcite{achen2002}.

\subsection*{Week 5: \classweek{4} -- Matching Methods I}
\begin{itemize}[leftmargin=*]
  \item Exact matching
  \item Propensity score and Mahalanobis score matching
  \item Assessing balance
  \item Sensitivity analysis
\end{itemize}
\textbf{Required:} \textcite{rosenbaum2017}, pp.\ 65--90; \textcite{rosenbaum2020}; \textcite{hansenbowers2008}.\\
\textbf{Recommended:} \textcite{rosenbaum2002a}, Ch.\ 3, \S 3.1--3.2, 3.4--3.5; \textcite{bifulco2012}; \textcite{arceneauxetal2010}.\\
\textbf{Application:} \textcite{cerdaetal2012}.

\subsection*{Week 6: \classweek{5} -- Matching Methods II}
\begin{itemize}[leftmargin=*]
  \item Weighting methods (IPW, AIPW)
  \item Entropy balancing
  \item Synthetic control methods
\end{itemize}
\textbf{Required:} \textcite{rosenbaum2017}, pp.\ 90--96; \textcite{hansen2011}; \textcite{chattopadhyayetal2020}.\\
\textbf{Recommended:} \textcite{rubin1979}; \textcite{hoetal2007}; \textcite{hainmueller2012}; \textcite{zubizarreta2015}.

\subsection*{Week 7: \classweek{6} -- Regression and Causal Inference}
\begin{itemize}[leftmargin=*]
  \item Regression as matching
  \item Parametric vs. non-parametric approaches
  \item Fixed effects models
  \item The debate over regression adjustments
\end{itemize}
\textbf{Required:} \textcite{gerbergreen2012}, Ch.\ 4; \textcite{rosenbaum2002c}; \textcite{lin2013}; \textcite{berk2004}, Ch. 5. 
.\\
\textbf{Recommended:} \textcite{freedman2008a}; \textcite{freedman2008b}; \textcite{freedman2008c}; \textcite{samii2012}; \textcite{abadieetal2020}.

\subsection*{Week 8: \classweek{7} -- Instrumental Variables I}
\begin{itemize}[leftmargin=*]
  \item The instrumental variables approach
  \item Identifying assumptions (relevance and exclusion)
  \item Two-stage least squares (2SLS)
  \item Weak instruments problem
\end{itemize}
\textbf{Required:} \textcite{gerbergreen2012}, Ch.\ 5--6; \textcite{rosenbaum2010}, \S 5.3; \textcite{rosenbaum1996}.\\
\textbf{Recommended:} \textcite{angristetal1996}; \textcite{imbensrosenbaum2005}.\\
\textbf{Application:} \textcite{albertsonlawrence2009}.

% SPRING BREAK: Week 8 (offset 8) is skipped -- no class on \SetDate[\firstclassdate]\AdvanceDate[56]\classformat\today

\subsection*{Week 9: \classweek{9} -- Instrumental Variables II}
\begin{itemize}[leftmargin=*]
  \item Local average treatment effects (LATE)
  \item Testing IV assumptions
  \item Applications: natural experiments in political science
  \item Encouragement designs
\end{itemize}
\textbf{Required:} \textcite{gerbergreen2012}, Ch.\ 7; \textcite{angristetal1996}.\\
\textbf{Recommended:} \textcite{rosenbaum2002c}, \S 2.3; \textcite{hansenbowers2008}; \textcite{kangetal2018}.

\subsection*{Week 10: \classweek{10} -- Regression Discontinuity Designs}
\begin{itemize}[leftmargin=*]
  \item Sharp vs. fuzzy RD designs
  \item Identification assumptions
  \item Bandwidth selection and specification tests
  \item Applications: electoral thresholds and policy discontinuities
\end{itemize}
\textbf{Required:} \textcite{cattaneoetal2020}; \textcite{caugheysekhon2011}; \textcite{imbenslemieux2008}.\\
\textbf{Recommended:} \textcite{lee2008}; \textcite{gelmanimbens2019}; \textcite{mccrary2008}; \textcite{saleshansen2020}; \textcite{angristpischke2008}, Ch.\ 6.

\subsection*{Week 11: \classweek{11} -- Difference-in-Differences}
\begin{itemize}[leftmargin=*]
  \item The parallel trends assumption
  \item Two-way fixed effects models
  \item Event studies and pre-trends testing
  \item Recent developments: staggered adoption and heterogeneous effects
\end{itemize}
\textbf{Required:} \textcite{rosenbaum2017}, pp.\ 162--167; \textcite{gerbergreen2012}, \S 4.1; \textcite{angristpischke2008}, Ch.\ 5.\\
\textbf{Recommended:} \textcite{lechner2011}; \textcite{manskipepper2018}; \textcite{dingli2019}; \textcite{imaikim2021}.

\subsection*{Week 12: \classweek{12} -- Panel Data and Fixed Effects}
\begin{itemize}[leftmargin=*]
  \item Panel data methods for causal inference
  \item Individual and time fixed effects
  \item First differences and within estimators
  \item Dynamic panel models
\end{itemize}
\textbf{Required:} \textcite{imaikim2021}; \textcite{zubizarretakeele2017}.\\
\textbf{Recommended:} \textcite{yangetal2016}; \textcite{pimenteletal2018}; \textcite{savjehigginssekhon2021}.

\subsection*{Week 13: \classweek{13} -- Directed Acyclic Graphs (DAGs)}
\begin{itemize}[leftmargin=*]
  \item Graphical representation of causal models
  \item Confounders, mediators, and colliders
  \item d-separation and conditional independence
  \item Using DAGs to guide analysis strategies
\end{itemize}
\textbf{Required:} \textcite{morganwinship2015}, Ch.\ 3; \textcite{pearl2009}, Ch.\ 1--2.\\
\textbf{Recommended:} \textcite{elwert2013}; \textcite{cinellietal2022}.

\subsection*{Week 14: \classweek{14} -- Mediation and Spillover Effects}
\begin{itemize}[leftmargin=*]
  \item Direct and indirect effects
  \item Mediation analysis
  \item Interference and spillover effects
  \item Network experiments
\end{itemize}
\textbf{Required:} \textcite{rosenbaum2007a}; \textcite{bowersetal2013}; \textcite{aronowsamii2017}.\\
\textbf{Recommended:} \textcite{manski2013}; \textcite{atheyecklesimbens2018}; \textcite{imaietal2011}.

\subsection*{Week 15: \classweek{15} -- Advanced Topics and Frontiers}
\begin{itemize}[leftmargin=*]
  \item Machine learning for causal inference
  \item Double/debiased machine learning
  \item Text as treatment and outcome
  \item Causal inference with qualitative data
\end{itemize}
\textbf{Required:} \textcite{atheyimbens2017}; \textcite{chernozhukovetal2018}.\\
\textbf{Recommended:} \textcite{wagerathey2018}; \textcite{egamietal2018}; \textcite{feder2022}.

\subsection*{Week 16: \classweek{16} -- Student Presentations}
\begin{itemize}[leftmargin=*]
  \item Final project presentations
  \item Course wrap-up and discussion
\end{itemize}



\vspace{1cm}

\begin{center}
\color{lightgray}\small
This syllabus is subject to change. Any modifications will be announced in class and posted on Canvas.
\end{center}

\clearpage
\section*{\faBook\ References}
\printbibliography[heading=none]

\end{document}
