\documentclass[11pt,letterpaper]{article}

% Modern packages for styling
\usepackage[utf8]{inputenc}
\usepackage[T1]{fontenc}
%\usepackage{helvet} % Helvetica font (sans serif)
%\usepackage[sfdefault]{FiraSans}
\usepackage[default]{lato}
\renewcommand{\familydefault}{\sfdefault} % Set sans serif as default
\usepackage[margin=1in]{geometry}
\usepackage{xcolor}
\usepackage{fontawesome5}
\usepackage{hyperref}
\usepackage{enumitem}
\usepackage{titlesec}
\usepackage{fancyhdr}
\usepackage{datetime}
\usepackage{advdate}

% Bibliography setup
\usepackage[style=authoryear, backend=biber, maxbibnames=10, maxcitenames=2]{biblatex}
\addbibresource{syllabus.bib}

% Define color scheme
\definecolor{primarycolor}{RGB}{19, 41, 75} % Illinois Navy Blue
\definecolor{accentcolor}{RGB}{232, 74, 39} % Illinois Orange
\definecolor{lightgray}{RGB}{100, 100, 100}

% Hyperlink setup
\hypersetup{
    colorlinks=true,
    linkcolor=primarycolor,
    urlcolor=accentcolor,
    citecolor=primarycolor
}

% Section title formatting
\titleformat{\section}
{\color{primarycolor}\Large\bfseries}
{\thesection}{1em}{}[\titlerule]

\titleformat{\subsection}
{\color{accentcolor}\large\bfseries}
{\thesubsection}{1em}{}

% Header and footer
\pagestyle{fancy}
\fancyhf{}
\fancyhead[L]{\color{lightgray}\small POLS 531: Causal Inference}
\fancyhead[R]{\color{lightgray}\small Spring 2026}
\fancyfoot[C]{\thepage}
\fancyfoot[R]{\footnotesize{Version~of~\today~\currenttime}}
\renewcommand{\headrulewidth}{0.4pt}
\renewcommand{\footrulewidth}{0pt}

% Date calculation setup
% CUSTOMIZE: Change this date to adjust the first day of class
\newcommand{\firstclassdate}{20/01/2026}

\newdateformat{classformat}{\THEDAY\ \monthname[\THEMONTH] \THEYEAR}

% Counter for automatic week numbering
\newcounter{weeknum}
\setcounter{weeknum}{0}

% \classweekdate{offset} - Outputs date that is 'offset' weeks from firstclassdate (no counter increment)
\newcommand{\classweekdate}[1]{%
  \SetDate[\firstclassdate]%
  \AdvanceDate[\the\numexpr#1*7\relax]%
  \classformat\today%
}

% \classweek{offset} - Prints the date that is 'offset' weeks from firstclassdate (increments weeknum)
% Example: \classweek{0} = first class, \classweek{1} = second week, \classweek{9} = week after spring break
\newcommand{\classweek}[1]{%
  \stepcounter{weeknum}%
  \classweekdate{#1}%
}

% \weekheading{title} - Creates subsection with auto-numbered week, date, and title
% Example: \weekheading{Introduction} -> "Week 1: 20 January 2026 -- Introduction"
% Week numbers and dates auto-increment together. Just add weeks in order.
% For spring break, simply add: \weekheading{No Class --- Spring Break}
% Adds PDF bookmark for navigation in PDF viewers.
\newcommand{\weekheading}[1]{%
  \stepcounter{weeknum}%
  \pdfbookmark[2]{Week \theweeknum: #1}{week\theweeknum}%
  \subsection*{Week \theweeknum: \classweekdate{\numexpr\theweeknum-1\relax} -- #1}%
}

% Legacy command for backwards compatibility (auto-incrementing)
% \weekdate - Automatically advances by 7 days each call
\newcounter{autoweek}
\setcounter{autoweek}{-1}
\newcommand{\weekdate}{%
  \stepcounter{autoweek}%
  \classweek{\theautoweek}%
}

\usepackage{parskip}

\begin{document}

% Custom title
\begin{center}
  {\Large\color{accentcolor}\bfseries Causal Inference for Political Science}\\
  {\Large\color{primarycolor}\bfseries POLS 531}\\
  {\large Spring 2026}\\
  {\normalsize University of Illinois at Urbana-Champaign}
  \rule{\textwidth}{1pt}
\end{center}

\pdfbookmark[1]{Course Information}{courseinfo}
\section*{\faInfoCircle\ Course Information}

\noindent
\begin{tabular}{@{}ll}
\textbf{\faCalendar\ Meeting Time:} & Tuesday, 9:00 AM -- 11:50 AM \\
\textbf{\faMapMarker\ Location:} & David Kinley Hall 314 \\
\textbf{\faUser\ Instructor:} & Jake Bowers \\
\textbf{\faEnvelope\ Instructor Email:} & \href{mailto:jwbowers@illinois.edu}{jwbowers@illinois.edu} \\
\textbf{\faClock\ Office Hours:} & by appointment at \href{https://calendly.com/jakebowers}{Calendly}\\
\textbf{\faBuilding\ Office:} & 432 David Kinley Hall \\
    \textbf{\faUser\ Methods Preceptor:} & Liliana Brock \\
    \textbf{\faEnvelope\ Methods Preceptor Email:} & \href{lcbrock2@illinois.edu}{lcbrock2@illinois.edu}
\end{tabular}

\pdfbookmark[1]{Overview}{overview}
\section*{\faBook\ Overview}

We infer what we cannot observe. Last term you engaged with the problems of
choosing better or worse descriptions of what we actually observe in data, and
also with the problems of statistical inference --- where we discussed better
and worse ways to learn about descriptions of phenomenon that we cannot
directly observe using tools like estimators and tests. Whether an estimator or
a test was a good one or bad one depended on assumptions that we had to make
about the process of observation. For example, we learned that a randomized
experiment and/or a random sample allows us to trust that our estimators are
unbiased and tests have controlled false positive error rates.

This term we dive deeper in the different approaches that statisticians and
methodologists have developed to make causal inferences.\footnote{I encourage
you to check out classes on sampling in order to go deeper into population
inference and classes on measurement and psychometrics in order to go deeper
into measurement inference.} When we say "causal inference" in this course we
are really talking about making statistical inferences that we feel comfortable
interpreting as reflecting counterfactual causal effects.\footnote{We recommend
you checkout \fullcite{waldner2025book} to learn more about causal inference that
does not involve statistical inference directly.}

What we will see is that ``statistical inference for counterfactual causal
quantities" aka ``causal inference" requires more assumptions than statistical
inference alone. The fact that we have to make assumptions is not in and of
itself a problem. However, more assumptions require more scrutiny. By the end
of the course, I hope that you feel confident choosing, using and scrutinizing
assumptions to enable you to make causal inferences from data and designs that
help you advance understanding of politics and society in your substantive
work.

\pdfbookmark[1]{Learning Objectives}{objectives}
\section*{\faBullseye\ Learning Objectives}

By the end of this course, students will be able to:

\begin{itemize}[leftmargin=*]

  \item Articulate the fundamental problem of causal inference and distinguish
    causal from descriptive research questions

  \item Design and analyze basic randomized experiments, including power
    analysis.

  \item Understand, implement, interpret, and evaluate causal inferences that
    depend on adjusting for confounders (including matching and parametric
    adjustment)
  
  \item Understand, implement, interpret, and evaluate causal inferences that
    depend on instrumental variables and other discontinuities (RDD)

  \item Understand, implement, interpret, and evaluate causal inferences that
    depend on assumptions about how outcomes change over time in panel data.

\end{itemize}

    To these ends I have designed a series of activities that should (1)
    give you opportunities to practice working with data and reasoning
    about statistics and (2) raise questions for discussion each week.

    \textbf{I do not lecture.} Rather, each week we will meet to engage with the
    questions that you have.

    \pdfbookmark[2]{Explorations}{explorations}
    \subsection*{Explorations}

    Every week or so, I will ask you to complete a short assignment that
    encourages you to engage creatively with the topics of that section of the
    course. I anticipate that you will work on most of these assignments in
    groups and a few alone  and that each of you will come to class prepared to
    discuss them.  I don't think that the groups should have more than 3 people
    in them. However, I'm willing to have larger groups if you talk with me
    about it. The point of the explorations is for you to (1) practice learning
    on your own (making mistakes, confronting confusing error messages, finding
    help online and elsewhere) and in a group (this is how you will learn about
    statistics for the rest of your career, so these explorations are supposed
    to help you to practice it now), (2) engage with the topic of the week so
    that you are prepared to come to class with questions and ideas, (3)
    practice coding and confronting coding errors.

    \pdfbookmark[2]{Final projects}{finalprojects}
    \subsection*{Final projects}

    The final project for this class allows you to apply one or more of the
    specific approaches from the class to a paper that is substantively
    interesting to you. You may also propose a different final project. The
    point of the final project is for you to show (to yourself and to the
    instructor) that you are on the path of mastery of this material.

    You are allowed to do this work with a co-author or alone as you see fit.

    I will be providing more detailed guidance about this assignment as the
    term goes on. At the moment I envision this project as a kind of
    \textbf{detailed pedagogical appendix to a paper} that you are writing
    and/or a \textbf{chapter in a methods book} in which you \textbf{teach} a
    reader about a given technique of causal inference and apply this technique
    to data and interpret the substantive meaning of the results of the
    analysis. This would involve (1) explaining the abstract assumptions about
    research design and outcomes and relationships among units and causal
    mechanisms that underly the statistical inference and the causal effects
    (ex. ``identification conditions''), (2) presenting evidence that helps the
    reader reason about the credibility of each and every assumption, (3)
    discussing how departures / failures of assumptions might encourage
    misleading interpretations of the results, (4) (maybe) propose, implement,
    and diagnose alternative approaches with different assumptions, (5) explain
    the version of sensitivity analysis that might help with questions about
    assumptions that cannot be easily addressed with data and implement and
    interpret that sensitivity analysis, (6) provide an overarching summary of
    the causal effect as far as you can tell given your work.

    So, this project will encourage you to go in-depth into one or perhaps two
    approaches to causal inference.

    \pdfbookmark[2]{Quizzes}{quizzes}
    \subsection*{In-class quizzes}

    We will probably begin each class with an ungraded in-class quizz about the
    material of the \textbf{previous week} so that each class reviews the
    previous week and also engages with new material. This should enable us to
    talk about each important topic at least twice before moving onto new
    material.

    \pdfbookmark[2]{Teaching sessions}{teaching_sessions}
    \subsection*{Teaching a topic}

    I ask that each student in the class take over the class for one 30 minute
    segment after the first session. You should use that time to teach the
    material for that week. This will enable you to get a head start on your
    final project (which will tend to focus on a single topic) by diving more
    deeply into the material than you might otherwise have done. You can give a
    15 min lecture and entertain questions from the group, assign an
    application of the topic as reading to the class and lead a discussion of
    the method as applied, invent an in-class activity, etc. I ask that you
    tell me your plans at least a week before the class session.

    \pdfbookmark[2]{My Expectations}{expectations}
    \subsection*{My Expectations}

    \begin{enumerate}

        \item I assume you are eager to learn. Eagerness, curiosity and
            excitement will impel your energetic engagement with the class
            throughout the term. If you are bored, not curious, or unhappy
            about the class you should come and talk with me immediately.
            Energetic engagement manifests itself in meeting with your
            classmates outside of the class, in asking questions during the
            class, and in taking the assignments seriously.

        \item I assume you are ready to work. Learning requires work. As much as
            possible I will encourage you to link practice directly to application
            rather than merely as a opportunity for me to rank you among your peers.
            Making work about learning rather than ranking, however, will make our work
            that much more difficult and time consuming. You will make errors. These
            errors are opportunities for you to learn --- some of your learning will be
            about how to help yourself and some will be about statistics. If you have
            too much to do this term consider dropping the course. Graduate
            school is a place for you to develop and begin to pursue your own
            intellectual agendas: this course may be important for you this term, or it
            may not. That is up for you to decide.

        \item  I assume some previous engagement with high school mathematics.

        \item You should ask questions when you don't understand things;
            chances are you're not alone.  \textbf{This class is an opportunity
            to practice courage:} I expect you to make a guess when I ask a
            question (in writing or in person), I expect that you will ask a
            question when you have a problem.

        \item \textbf{Do the work.} This does not mean divide the work up among
            your classmates so that you only do part of the work. Each person
            should engage with all of the work even if the people who writes it
            up changes from week to week.

        \item 	All papers written in this class will assume familiarity with
            the principles of good writing in \fullcite{beck:1986}.

        \item 	All final written work will be turned in as pdf files unless we have another
            specific arrangement.\footnote{For example, if you have some reason why pdf
                files make your life especially difficult, then of course I will work with
            you find another format.} I will not accept Microsoft, Apple, or any other
            proprietary format.
    \end{enumerate}

    \pdfbookmark[2]{Late Work}{latework}
    \subsection*{Late Work}
    I do not like evaluation for the sake of evaluation. Evaluation should
    provide opportunities for learning. So, if you'd prefer to spend
    more time using the final project in this class to learn more, I am
    happy for you to take that time. I will not, however, entertain late
    submissions for any subsidiary assignment that are due
    throughout the term. If you think that you and/or the rest of the
    class have a compelling reason to change the due date on one of those
    assignments, let me know in advance and I will probably just change
    the due date for the whole class.

    \pdfbookmark[2]{Incompletes}{incompletes}
    \subsection*{Incompletes} Incomplete grades at the end of the term are fine
    in theory but terrible in practice. I urge you to avoid an incomplete in
    this class. If you must take an incomplete, you must give me \emph{at
    least} 2 months from the time of turning in an incomplete before you can
    expect a grade from me and it may well take me much longer. This means that
    if your fellowship, immigration status, or job depends on erasing an
    incomplete in this class, you should not leave this incomplete until the
    last minute.

    \pdfbookmark[2]{Grades are Feedback}{grades}
    \subsection*{Grades are Feedback}

    Humans need feedback to close the gap between  intention and action. They
    also need feedback to feel good about their progress and to motivate them.
    In this class I will use grades as feedback. All grades except for the
    final grade will be satisfactory, unsatisfactory (with the possibility of
    "outstanding"), and fail. These map roughly onto  A+=outstanding,
    A=satisfactory, C=unsatisfactory, and F=fail (i.e. you didn't try).

    I'll calculate your grade for the course this way: 30\% explorations (when
    you turn it in as a group everyone in the group receives the same grade,
    satisfactory if you are creative and thoughtful and diligent,
    unsatisfactory if you are not or if you don't seem to be getting the
    concepts, no late work accepted); 40\% final project (graded using a rubric
    to help you not forget important topics); 20\% in-class teaching (sat or
    unsat); 10\% attendance (satisfactory if you show up, fail if not). 

    You can miss two classes without grade penalty.

    I will drop your lowest exploration grade as well.

    Because moments of evaluation are also moments of learning in this
    class, I do not curve. If you all perform at 100\%, then I will give
    you all As.

    You can redo any  evaluation or the final paper in order  to  increase
    your grade on that  assignment. If you want to resubmit something already
    graded, you need  to let me know  in advance so that  I  can make  time to
    grade it again. If you want to resubmit work after the end of the term,
    that is also ok, but I may take many months to grade that work.

    \pdfbookmark[2]{Books}{books}
    \subsection*{Books}

    No book is perfect for all students. I suggest you ask around, look at
    other syllabi online, and just browse the shelves at the library and used
    bookstores to find books that make things clear to you.  I will be adding
    some recommendations here. Let me know now if you have favorites.

    If you discover any books or websites that are particularly useful to you,
    please alert me and the rest of the class about them. Thanks!

    \pdfbookmark[2]{Academic Integrity}{integrity}
    \subsection*{Academic Integrity} According to the Student Code, `It is
    the responsibility of each student to refrain from infractions of
    academic integrity, from conduct that may lead to suspicion of such
    infractions, and from conduct that aids others in such infractions.’
    Please know that it is my responsibility as an instructor to uphold the
    academic integrity policy of the University, which can be found here:
    \url{http://studentcode.illinois.edu/article1_part4_1-401.html}.

\pdfbookmark[2]{Accessibility}{accessibility}
\subsection*{Accessibility}
Students with disabilities who require accommodations should contact Disability Resources and Educational Services (DRES) and provide the instructor with accommodation letters as soon as possible.

\pdfbookmark[2]{Mental Health}{mentalhealth}
\subsection*{Mental Health}
The University of Illinois offers a range of mental health services. If you are feeling overwhelmed, anxious, or depressed, please reach out to the Counseling Center (\href{https://www.counselingcenter.illinois.edu/}{counselingcenter.illinois.edu}) or call the Consultation and Crisis Line at 217-333-3704.

\pdfbookmark[2]{Diversity and Inclusion}{diversity}
\subsection*{Diversity and Inclusion}
The Political Science Department is committed to building an inclusive environment where all students feel welcomed and supported. We value diverse perspectives and encourage respectful dialogue.

\pdfbookmark[1]{Course Schedule}{schedule}
\section*{\faIcon{calendar-alt}\ Course Schedule}

    \textbf{Note: } This schedule is preliminary and subject to change. If
    you miss a class make sure you contact me or one of your colleagues to
    find out about changes.


    \textbf{Data: } I'll be bringing in data that I have on hand. This
    means our units of analysis will often be individual people or perhaps
    political or geographic units, mostly in the United States. I'd love
    to use other data, so feel free to suggest and provide it to me ---
    come to office hours and we can talk about how to use your favorite
    datasets in the class.

% HOW TO USE: \weekheading{title} creates "Week N: [date] -- title"
% Week numbers and dates auto-increment. Just list weeks in order.
% For spring break: \weekheading{No Class --- Spring Break}
% To reorder topics: just move the entire \weekheading block; week numbers adjust automatically

\weekheading{Introduction to the class, the Fundamental Problem of Causal Inference, Working with Data}
\begin{itemize}[leftmargin=*]
  \item The fundamental problem of causal inference and the potential outcomes framework
  \item Causal effects as functions of potential outcomes including individual causal effects and average treatment effects (ATE, ATT, ATC)
  \item Directed Acyclic Graphs (DAGs) to help us reason and communicate about causal relationships: the Backdoor criterion and colliders
  \item Workflow for credibility
\end{itemize}

\textbf{Required:}
\begin{enumerate}[noitemsep,leftmargin=*]
  \item \fullcite{holland1986}
  \item \fullcite{rosenbaum2017}, Ch. 1--2
  \item \fullcite{morganwinship2015}, Ch. 3
  \item \fullcite{cunningham2021causal} Ch. 3--4
  \item \fullcite{bowers2016future}
  \item \fullcite{bowers2022latex} (no need to do the exercises)
\end{enumerate}

\textbf{Recommended:}
\begin{enumerate}[noitemsep,leftmargin=*]
  \item \fullcite{waldner2025book} Ch. 2 up through \S 2.2.2 on DAGs
  \item \fullcite{bellemare2024frontdoor} on front-door adjustment
  \item \fullcite{glynn2018front} on front-door adjustment
\end{enumerate}

\weekheading{Randomized Experiments: Randomization inference and Identification}
\begin{itemize}[leftmargin=*]
  
  \item Randomization as the ``reasoned basis for [statistical] inference"
      about causal effects via hypothesis testing and estimation of
        average causal effects.

   \item How does randomization \textbf{identify} a causal effect? (What do
       people mean when they say ``identification" in the context of causal
        inference?)

\end{itemize}

\textbf{Required:}
\begin{enumerate}[noitemsep,leftmargin=*]
  \item \fullcite{rosenbaum2017}, Ch. 3
  \item \fullcite{fisher1935a}, Ch. 2
  \item \fullcite{bowersleavitt2020}
  \item \fullcite{hernanrobins2020} (the online 2025 version), Ch. 1--3
\end{enumerate}
\textbf{Recommended:}
\begin{enumerate}[noitemsep,leftmargin=*]
  \item \fullcite{rosenbaum2002a}, Ch. 2
  \item \fullcite{rosenbaum2020book}, Ch. 2
\end{enumerate}

\weekheading{Randomized Experiments: Practicalities and design}
\begin{itemize}[leftmargin=*]
  \item Power analysis and sample size calculations
  \item Types of random assignment: simple, complete, blocked/stratified, clustered
\end{itemize}
\textbf{Required:}
\begin{enumerate}[noitemsep,leftmargin=*]
  \item \fullcite{gerbergreen2012}, Ch. 2--4
  \item \fullcite{bowersVoorsIchino2021book} Ch. 4 and 7
\end{enumerate}
\textbf{Recommended:}
\begin{enumerate}[noitemsep,leftmargin=*]
  \item \fullcite{rosenbaum2002c}
  \item \fullcite{freedman2008a}
  \item \fullcite{freedman2008b}
  \item \fullcite{freedman2008c}
  \item \fullcite{lin2013}
  \item \fullcite{aronowmiddleton2013}
  \item \fullcite{middletonaronow2015}
  \item \fullcite{miratrixetal2013}
\end{enumerate}

\weekheading{Unconfoundedness assumptions and bipartite stratified adjustment}
\begin{itemize}[leftmargin=*]
  \item The conditional independence assumption or "selection on observables"
  \item Exact matching, Matching on scalars, on scores (propensity and Mahalanobis scores)
  \item Assessing and justifying stratified designs (balance as assessed by the omnibus test; substantive comparability)
  \item Sensitivity analysis
\end{itemize}
\textbf{Required:}
\begin{enumerate}[noitemsep,leftmargin=*]
    \item \fullcite{rosenbaum2020}
    \item \fullcite{rosenbaum2020}, Ch. 8--10, 14
  \item \fullcite{hansenbowers2008}
  \item \fullcite{bindrubin2019}
  \item TBA Miratrix and Leavitt on Matching
\end{enumerate}
\textbf{Recommended:}
\begin{enumerate}[noitemsep,leftmargin=*]
  \item \fullcite{rosenbaum2017}, Ch.\ 11
\end{enumerate}

\weekheading{Unconfoundedness assumptions and non-bipartite stratified adjustment}
\begin{itemize}[leftmargin=*]
  \item Matching on continuous or non-binary ``treatments''
\end{itemize}
\textbf{Required:}
\begin{enumerate}[noitemsep,leftmargin=*]
  \item \fullcite{rosenbaum2020book}, Ch. 12, 14
\end{enumerate}
\textbf{Recommended:}
\begin{enumerate}[noitemsep,leftmargin=*]
  \item \fullcite{rabb2022pnas}
  \item \fullcite{wong2025maps}
\end{enumerate}

\weekheading{No Class --- reschedule}

\weekheading{Unconfoundedness assumptions and parametric adjustment}
\begin{itemize}[leftmargin=*]
  \item Regression as matching
  \item Parametric vs. non-parametric approaches
  \item Fixed effects models
  \item The debate over regression adjustments
\end{itemize}
\textbf{Required:}
\begin{enumerate}[noitemsep,leftmargin=*]
  \item \fullcite{gelmanhill2006}, \S 9.0--9.2
  \item \fullcite{berk2010}
  \item \fullcite{achen2002}
  \item \fullcite{berk2004}, Ch. 5
\end{enumerate}
\textbf{Recommended:}
\begin{enumerate}[noitemsep,leftmargin=*]
  \item \fullcite{samii2012}
  \item \fullcite{abadieetal2020}
\end{enumerate}

\weekheading{Instrumental Variables designs}
\begin{itemize}[leftmargin=*]
  \item The instrumental variables approach
  \item Identifying assumptions (ignorable treatment, SUTVA, relevance, monotonicity, exclusion)
  \item Estimating the LATE using (1) 2SLS, (2) Placebo arms and testing the null hypothesis of no effects effects on compliers.
\end{itemize}
\textbf{Required:}
\begin{enumerate}[noitemsep,leftmargin=*]
  \item \fullcite{angristetal1996}
  \item \fullcite{rosenbaum1996}
  \item \fullcite{gerbergreen2012}, Ch. 5--6
  \item \fullcite{rosenbaum2020book}, Ch. 5
  \item \fullcite{rosenbaum2017}, Ch. 13 
  \item \fullcite{sovey2011instrumental}
\end{enumerate}
\textbf{Recommended:}
\begin{enumerate}[noitemsep,leftmargin=*]
  \item \fullcite{imbensrosenbaum2005}
  \item \fullcite{hansenbowers2009}
\end{enumerate}

% SPRING BREAK: Week 8 (offset 8) is skipped -- no class on \SetDate[\firstclassdate]\AdvanceDate[56]\classformat\today

\weekheading{No Class --- Spring Break}

\weekheading{Discontinuities as instruments and natural experiments}
\begin{itemize}[leftmargin=*]
    \item Sharp vs. fuzzy RD designs; relationship between ``regression discontinuity designs'' (RDD) and instrumental variables (IV) and ``natural experiments'' (unconfoundedness assumptions versus instrumental variables style assumptions)
  \item Identification assumptions
  \item Bandwidth selection and specification tests
\end{itemize}
\textbf{Required:}
\begin{enumerate}[noitemsep,leftmargin=*]
  \item \fullcite{cattaneoetal2020}
  \item \fullcite{caugheysekhon2011}
  \item \fullcite{imbenslemieux2008}
\end{enumerate}
\textbf{Recommended:}
\begin{enumerate}[noitemsep,leftmargin=*]
  \item \fullcite{lee2008}
  \item \fullcite{gelmanimbens2019}
  \item \fullcite{mccrary2008}
  \item \fullcite{saleshansen2020}
  \item \fullcite{angristpischke2008}, Ch.\ 6
\end{enumerate}

\weekheading{Panel and Multilevel Data: Using unconfoundedness via risk-set matching and weighting }
\begin{itemize}[leftmargin=*]
  \item Making and strengthening arguments for unconfoundedness using stratification with longitudinal/panel data
  \item Multilevel matching (when interventions are at one level (like schools) and outcomes are measured at another level (like students))
\end{itemize}
\textbf{Required:}
\begin{enumerate}[noitemsep,leftmargin=*]
  \item \fullcite{rosenbaum2020book}, Ch. 13
  \item \fullcite{rosenbaum2017}, pp. 227--229
  \item \fullcite{zubizarretakeele2017}
  \item \fullcite{pimenteletal2018}
\end{enumerate}
\textbf{Recommended:} 

\weekheading{Panel Data 1: Difference-in-Differences using assumptions about outcomes }
\begin{itemize}[leftmargin=*]
  \item The parallel trends assumption
  \item Panel data methods for causal inference
  \item Individual and time fixed effects
  \item First differences and within estimators
  \item Dynamic panel models
  \item Event studies and pre-trends testing
  \item Recent developments: staggered adoption and heterogeneous effects
\end{itemize}
\textbf{Required:} 
\begin{enumerate}
    \item TBA on difference in differences
    \item \fullcite{imaikim2021}
\end{enumerate}
\textbf{Recommended:}
\begin{enumerate}[noitemsep,leftmargin=*]
    \item \fullcite{cunningham2021causal}, Ch. 9
  \item \fullcite{glynnkashin2017}
  \item TBA something from Leavitt and/or Laura Hatfield \url{https://diff.healthpolicydatascience.org/}
\end{enumerate}

\weekheading{Panel Data 3: Combining weighting and assumptions about outcomes}
\begin{itemize}[leftmargin=*]
    \item When to use Two-way fixed effects and when to avoid them (or when to add weights)
  \item Dynamic panel models
\end{itemize}
\textbf{Required:}
\begin{enumerate}[noitemsep,leftmargin=*]
  \item \fullcite{shen2024anatomy}
  \item \fullcite{imaikim2021}
\end{enumerate}
\textbf{Recommended:} TBA

\weekheading{Panel Data 4: Synthetic controls via more assumptions about outcomes}
\begin{itemize}[leftmargin=*]
    \item Synthetic controls with one treated unit.
    \item More generalized approaches
\end{itemize}
\textbf{Required:}
\begin{enumerate}[leftmargin=*]
  \item TBA
\end{enumerate}
\textbf{Recommended:}
\begin{enumerate}[leftmargin=*]
  \item \fullcite{abadieetal2012}
  \item \fullcite{abadie2020}
  \item \fullcite{ben-michaeletal2021}
  \item \fullcite{xu2017generalized}
\end{enumerate}

\weekheading{TBA to discuss with the class}
\textbf{Required:}
\begin{enumerate}[leftmargin=*]
  \item TBA
\end{enumerate}
\textbf{Recommended:}
\begin{enumerate}[leftmargin=*]
  \item TBA
\end{enumerate}


\weekheading{Final Questions}
\begin{itemize}[leftmargin=*]
  \item Each person brings a question 
  \item Course wrap-up and discussion
\end{itemize}


\pdfbookmark[2]{Other topics}{othertopics}
\subsection*{Other topics}

\begin{itemize}
    \item Contrasting assumptions about different kinds of unconfoundedness using multiple control groups and differential comparisons.
    \item Combining assumptions about treatment assignment and outcome processes: double robustness
    \item Techniques based on making arguments in favor of unconfoundedness assumptions: IPW, AIPW, Balancing weights.
    \item Techniques based on making arguments in favor of DAG structures: Proximal causal inference and Negative controls.
    \item Other estimands: Mediation and indirect effects, Average interference effects on networks and graphs
    \item Other hypotheses: About maximum effects, median effects; About patterns of interference effects on networks and graphs
        \begin{enumerate}[noitemsep]
          \item \fullcite{caughey2023randomisation}
          \item \fullcite{kim_su_bowers_li2025}
          \item \fullcite{bowersetal2013}
          \item \fullcite{bowersetal2016}
        \end{enumerate}
    \item Predictive Bayesian causal inference and machine learning.\\
        \textbf{Required:}
        \begin{enumerate}[noitemsep]
          \item \fullcite{atheyimbens2017}
          \item \fullcite{chernozhukovetal2018}
        \end{enumerate}
        \textbf{Recommended:}
        \begin{enumerate}[noitemsep]
          \item \fullcite{wagerathey2018}
          \item \fullcite{egamietal2018}
          \item \fullcite{feder2022}
        \end{enumerate}
    \item Causal inference via process tracing
\end{itemize}

\vspace{1cm}

\begin{center}
    \color{lightgray}\small
    This syllabus is subject to change. Any modifications will be announced in class and posted on Canvas.
\end{center}

\clearpage
\pdfbookmark[1]{References}{references}
\section*{\faBook\ References}
\printbibliography[heading=none]

\end{document}
